\documentclass{beamer}

\usepackage{proof}
\usepackage{ulem}
\usepackage{dashrule}

\title{Curry-Howard From The Ground Up}
\author{Michael Arntzenius}
%% TODO: \date

%\usetheme{Malmoe}
%\usecolortheme{seahorse}

\definecolor{ignore}{rgb}{0.8,0.8,0.8}

\newcommand{\isassumed}{\textsf{\tiny{\textit{(assumed)}}}}
\newcommand{\assumed}[1]{\deduce{#1}{\isassumed}}
\newcommand{\x}{\times}
\newcommand{\impl}{\supset}
\newcommand{\steps}{\mapsto}
\newcommand{\subst}[1]{[#1]\,}

\newcommand{\wedgeI}{\!\wedge\rn{I}}
\newcommand{\wedgeE}{\!\wedge\rn{E}}
\newcommand{\implI}{\!\impl\!\rn{I}}
\newcommand{\implE}{\!\impl\!\rn{E}}

\begin{document}


\begin{frame}
  \titlepage
\end{frame}

%% \begin{frame}
%%   \frametitle{Table of Contents}
%%   \tableofcontents
%% \end{frame}

\begin{frame}
  \frametitle{What's this talk about?}
  \begin{enumerate}
  \item A simple formal logic (natural deduction)
  \item A simple programming language ($\lambda$-calculus)
  \item How they're \textit{secretly the same thing!!}
    %% TODO: more convincing/accurate
  \item Where is this relevant? % (``why should I care?'')
  \end{enumerate}
\end{frame}


\section{1. Formal Logic}
%\section{A simple formal logic (natural deduction)}

%\begin{frame} \tableofcontents[currentsection] \end{frame}

\begin{frame}
  %\frametitle{What's a formal logic?}

  \begin{center}
    {\Large\bf What's formal logic?}
  \end{center}
  %Something that makes it \textbf{perfectly clear} when a proof is valid.

\end{frame}

\newcommand{\rn}[1]{\ensuremath{\mathtt{#1}}}

\begin{frame}
  \frametitle{Inference rules}

  % Inference rules are one way of structuring formal proofs.

  \[
  \infer[\rn{RuleName}]{conclusion}{premise_1 & premise_2 & \cdots & premise_n}
  \]

  \textbf{If} $premise_1$, $premise_2$, ..., and $premise_n$,\\
  \textbf{then} $conclusion$,\\
  \textbf{by} rule $\rn{RuleName}$.
\end{frame}

\begin{frame}
  \frametitle{Defining inference rules}

  % Inference rules can be \textbf{defined}:
  \[
  \infer[\rn{transitivity}]{a < c}{a < b & b < c}
  \]
  This \textbf{defines} a rule of inference called \rn{transitivity}.

\end{frame}

\begin{frame}
  \frametitle{Using inference rules to prove things}
  \vspace{-1.5em}
  \[\infer[\rn{transitivity}]{a < c}{a < b & b < c}\]
  \vspace{.2em}

  Suppose we know that $a < b$, $b < c$, and $c < d$. We'd like to
  \textbf{prove} that $a < d$ using our \rn{transitivity} rule.

  \onslide<2->{\[
    \infer[\rn{transitivity}]{a < d}{
      \assumed{a < b} &
      \infer[\rn{transitivity}]{b < d}{
        \assumed{b < c} &
        \assumed{c < d}}}
  \]}
\end{frame}

%% \begin{frame}
%%   \frametitle{Using inference rules in proofs}

%%   % Once defined, we can use inference rules in \textbf{proofs}:
%%   % Inference rules can be PLUGGED INTO one another.

%%   \[
%%   \infer[\rn{transitivity}]{a < d}{
%%     \assumed{a < b} &
%%     \infer[\rn{transitivity}]{b < d}{
%%       \assumed{b < c} &
%%       \assumed{c < d}}}
%%   \]

%%   This \textbf{proves} that $a < d$\\ \textbf{from} the (unproven) assumptions $a
%%   < b$, $b < c$, and $c < d$,\\ \textbf{using} the \rn{transitivity} rule.

%% \end{frame}

\begin{frame}
  \frametitle{A simple logic}
  \begin{center}
    \begin{tabular}{cc}
      Connective & Meaning\\
      \hline
      $A \wedge B$ & ``$A$ and $B$''\\
      $A \impl B$ & ``$A$ implies $B$'', or ``given $A$ then $B$''\\
      \color{ignore} $A \vee B$
      & \color{ignore} ``$A$ or $B$ (\textit{and I know which one})''\\
      \color{ignore} $\neg A$
      & \color{ignore} ``not $A$'' or ``refutation of $A$''\\
      %% \color{ignore} $\forall x.\, A$ & \color{ignore} ``$A$ holds of all $x$''\\
      %% \color{ignore} $\exists x.\, A$ & \color{ignore} ``$A$ holds of some $x$''\\
    \end{tabular}
  \end{center}
\end{frame}


%\subsection{The $\wedge$ connective}
\begin{frame}
  \frametitle{A simple logic: $\wedge$}

  %This rule lets you prove $A \wedge B$:
  \[
  \infer[\wedgeI]{A \wedge B}{A & B}
  \qquad
  %Given $A \wedge B$, these rules let you prove other things:
  \onslide<2->{\infer[\wedgeE_1]{A}{A \wedge B}}
  \qquad
  \onslide<3->{\infer[\wedgeE_2]{B}{A \wedge B}}
  \]
\end{frame}

\begin{frame}
  \frametitle{A simple logic: $\wedge$: example proofs}
  \[
  \onslide<2->{
    \infer[\wedgeE_1]{B}{
      \infer[\wedgeE_2]{B \wedge C}{
        \assumed{A \wedge (B \wedge C)}}
    }
  }
  \qquad
  \onslide<3->{
    \infer[\wedgeI]{A \wedge A}{\assumed{A} & \assumed{A}}
  }
  \qquad
  \onslide<4>{
    \infer[\wedgeI]{B \wedge C}{
      \infer[\wedgeE_2]{B}{\assumed{A \wedge B}}
      & \assumed{C}}
  }
  \]

  {\small For reference:
  \[
  \infer[\wedgeI]{A \wedge B}{A & B} \qquad
  \infer[\wedgeE_1]{A}{A \wedge B} \qquad
  \infer[\wedgeE_2]{B}{A \wedge B}
  \]
  }
\end{frame}


%\subsection{The $\impl$ connective}
\begin{frame}
  \frametitle{A simple logic: $\impl$}
  \[
  \infer[\implI]{A \impl B}{???}
  \qquad
  \infer[\implE]{B}{A \impl B & A}
  \]
\end{frame}

\begin{frame}
  %% omit this if people seem comfortable
  \frametitle{A simple logic: $\impl$: example proof (1)}

  Let's try to prove $C$ given $A$, $B$, and $(A \wedge B) \impl C$:
  \onslide<2>{\[
    \infer[\implE]{C}{
      \assumed{(A \wedge B) \impl C} &
      \infer[\wedgeI]{A \wedge B}{\assumed{A} & \assumed{B}}
    }
  \]}

  For reference:
  \[
  \infer[\implI]{A \impl B}{???}
  \qquad
  \infer[\implE]{B}{A \impl B & A}
  \]
\end{frame}

\newcommand{\lhyp}[3]{%
\deduce[\ \ \hdashrule{#1}{0.4pt}{2pt 1pt}_{#2}]{#3}{}%
}
%% \newcommand{\hyp}[2]{%
%% \deduce[\ \ \hdashrule{0.6cm}{0.4pt}{2pt 1pt}_{#1}]{#2}{}%
%% }
\newcommand{\hyp}[2]{\lhyp{0.6cm}{#1}{#2}}

\begin{frame}
  \frametitle{A simple logic: $\impl$}
  \[
  \infer[\implI\onslide<3->{_x}]{A \impl B}{
    \alt<2->{
      \deduce[\ ^{\vdots}]{B}{
        \deduce[\onslide<3->{\ \ \hdashrule{0.6cm}{0.4pt}{2pt 1pt}_x}]{A}{}
      }
    }{???}
  }
  \qquad
  \infer[\implE]{B}{A \impl B & A}
  \]
\end{frame}

\begin{frame}
  \frametitle{A simple logic: $\impl$: example proof (2)}

  Let's try to prove, given $A \impl B$, that $A \impl (A \wedge B)$:

  \onslide<2->{\[
  \infer[\implI_x]{A \impl (A \wedge B)}{
    \infer[\wedgeI]{A \wedge B}{
      \hyp{x}{A} &
      \infer[\implE]{B}{
        \assumed{A \impl B} &
        \hyp{x}{A}
      }
    }
  }
  \]}

  For reference:
  \[
  \infer[\implI_x]{A \impl B}{
    \deduce[\ ^{\vdots}]{B}{\hyp{x}{A}}}
  \qquad
  \infer[\implE]{B}{A \impl B & A}
  \]

\end{frame}

\begin{frame}
  \frametitle{A simple logic: $\impl$: example proof (3)}

  What's wrong with this ``proof'' of $(A \impl A) \wedge A$?
  \onslide<2->{\[
  \infer[\wedgeI]{(A \impl A) \wedge A}{
    \infer[\implI_x]{A \impl A}{
      \hyp{x}{A}
    }
    &
    \hyp{x}{A}
  }
  \]}

  For reference:
  \[
  \infer[\implI_x]{A \impl B}{
    \deduce[\ ^{\vdots}]{B}{\hyp{x}{A}}}
  \qquad
  \infer[\implE]{B}{A \impl B & A}
  \]
\end{frame}

\begin{frame}
  \frametitle{A simple logic: $\impl$: example proof (4)}
  %% omit if people feel comfortable

  Let's try to prove $(A \wedge B) \impl C$ from $A \impl (B \impl C)$:

  \onslide<2->{\[
  \infer[\implI_x]{(A \wedge B) \impl C}{
    \infer[\implE]{C}{
      \infer[\implE]{B \impl C}{
        \assumed{A \impl (B \impl C)} &
        \infer[\wedgeE_1]{A}{\lhyp{1.15cm}{x}{A \wedge B}}
      }
      &
      \infer[\wedgeE_2]{B}{\lhyp{1.15cm}{x}{A \wedge B}}
    }
  }
  \]}

  For reference:
  \[
  \infer[\implI_x]{A \impl B}{
    \deduce[\ ^{\vdots}]{B}{\hyp{x}{A}}}
  \qquad
  \infer[\implE]{B}{A \impl B & A}
  \]
\end{frame}


\begin{frame}
  \frametitle{A simple logic (all together now)}
  \[
  \infer[\wedgeI]{A \wedge B}{A & B}
  \qquad
  \infer[\wedgeE_1]{A}{A \wedge B}
  \qquad
  \infer[\wedgeE_2]{B}{A \wedge B}
  \]

  \[
  \infer[\implI_x]{A \impl B}{
    \deduce[\ ^{\vdots}]{B}{\hyp{x}{A}}}
  \qquad
  \infer[\implE]{B}{A \impl B & A}
  \]
\end{frame}


\section{2. Programming Language}
%\section{A simple programming language ($\lambda$-calculus)}

\begin{frame}
  %\frametitle{A $\lambda$-calculus primer}
  \begin{center}
    {\Large \bf A simple programming language \\
    ($\lambda$-calculus)}
  \end{center}
\end{frame}

\newcommand{\pair}[2]{\langle{#1},{#2}\rangle}
\newcommand{\projl}{\pi_1\,}
\newcommand{\projr}{\pi_2\,}

\begin{frame}
  \frametitle{A simple $\lambda$-calculus}
  %% x stands for variables
  %% a,b stands for expressions
  %% A,B stand for types
  \begin{center}
    \begin{tabular}{ccc}
      Syntax & What is it? & in Python\\
      \hline
      $\lambda x. a$
      & Anonymous function
      & \texttt{lambda\ x:\ a}\\
      $a\,b$
      & Function application
      & \texttt{a(b)}\\
      $\pair{a}{b}$ & Make a pair & \texttt{(a,b)}\\
      $\projl a$ & First element of pair & \texttt{a[0]}\\
      $\projr a$ & Second element of pair & \texttt{a[1]}\\
    \end{tabular}
  \end{center}
\end{frame}

\begin{frame}
  \begin{center}
    {\Large\bf What does $\projl (\lambda x. x)$ do?}

    \vspace{1em}
    {\large In Python: \texttt{(lambda\ x:\ x)[0]}}
  \end{center}
\end{frame}


\begin{frame}
  \frametitle{A simple $\lambda$-calculus: types}

  \begin{center}
    \begin{tabular}{cccc}
      Type & What is it? & in Haskell\\
      \hline
      $A \to B$ & Functions from $A$ to $B$ & \texttt{A -> B}\\
      $A \x B$ & Pairs of $A$s and $B$s & \texttt{(A,B)}\\
    \end{tabular}
  \end{center}
\end{frame}

\begin{frame}
  \begin{center}
    {\Large \bf How do I know what type an expression has?}

    \vspace{1em}
    \onslide<2>{\Large\bf Maybe we can we use inference rules!}
  \end{center}
\end{frame}

\newcommand{\isa}[2]{\visible<-5>{{ {#1} :\ }} {#2}}
\newcommand{\xname}[2]{\alt<7->{#2}{#1}}
\newcommand{\xx}{\alt<7->{\wedge}{\x}}
\newcommand{\xto}{\alt<7->{\impl}{\to}}

\begin{frame}
  \frametitle{A simple $\lambda$-calculus: typing rules}

  %We write $a : A$ for ``the expression $a$ has type $A$''.

  \[
  \onslide<2->{
    \infer[\xname{\rn{pair}}{\wedge\rn{I}}]{
      \isa{\pair{a}{b}}{A \xx B}
    }{
      \isa{a}{A} & \isa{b}{B}}
  }
  \qquad
  \onslide<3->{
    \infer[\xname{\rn{proj}}{\wedge\rn{E}}_1]{\isa{\projl a}{A}}{\isa{a}{A \xx B}}
  }
  \qquad
  \onslide<3->{
    \infer[\xname{\rn{proj}}{\wedge\rn{E}}_2]{\isa{\projr a}{B}}{\isa{a}{A \xx B}}
  }
  \]
  \[
  \onslide<4->{
    \infer[\xname{\rn{app}}{\implE}]{
      \isa{f\,a}{B}
    }{\isa{f}{A \xto B} & \isa{a}{A}}
  }
  \qquad
  \onslide<5->{
    \infer[\xname{\rn{lam}}{\implI}]{
      \isa{\lambda x. b}{A \xto B}
    }{
      %% this is an utter hack
      \deduce[\ ^{\vdots}]{\isa{b}{B}}{\isa{x}{A}}
    }
  }
  \]
\end{frame}


\section{3. Curry-Howard}
%\section{How they're {\it secretly the same thing!!}}

\begin{frame}
  \begin{center}
    {\Large\bf Our programming language\\ was a logic in disguise!}

    \only<2>{\Large\bf ... are expressions proofs in disguise?}
    % They sure are!
  \end{center}
\end{frame}

\begin{frame}
  \begin{center}
    \frametitle{Curry-Howard: programs as proofs}
    Here's a simple proof:
    \[
    \infer[\wedgeI]{B \wedge A}{
      \infer[\wedgeE_2]{B}{\assumed{A \wedge B}}
      &
      \infer[\wedgeE_1]{A}{\assumed{A \wedge B}}
    }
    \]
    This shows that $\wedge$ is commutative.

    % Let's turn it into a program.
  \end{center}
\end{frame}

\begin{frame}
  \begin{center}
    \frametitle{Curry-Howard: programs as proofs}
    Here's a simple \sout{proof} {\color{red} program}:
    \[
    \infer[\rn{pair}]{\pair{\projr a}{\projl a} : B \x A}{
      \infer[\rn{proj}_2]{\projr a : B}{a : A \x B}
      &
      \infer[\rn{proj}_1]{\projl a : A}{a : A \x B}
    }
    \]

    In Python: \texttt{(a[1], a[0])}.
    This swaps a pair.

    % Swapping a pair corresponds to commutativity of conjunction!
  \end{center}
\end{frame}

\begin{frame}
  \begin{center}
    {\Large\bf Commutativity of $\wedge$ \\=\\\vspace{4pt} Swapping a pair!}
  \end{center}
\end{frame}


\section{4. Further connections}
%\section{The bigger picture (``why should I care'')}

\begin{frame}
  \frametitle{What have we just learned?}
  \begin{center}
    \begin{tabular}{rcl}
      propositions &=& types\\
      proofs &=& expressions\\
      ``and'', $\wedge$ &=& pairs, $\x$\\
      ``implies'', $\impl$ &=& functions, $\to$\\
      rules of logic &=& rules of typing\\
      commutativity of $\wedge$ &=& swapping a pair
    \end{tabular}

    \vspace{1em}
    {\large\bf We have found a bridge between\\ logic and programming languages.}
  \end{center}
\end{frame}

\begin{frame}
  \frametitle{What use is this?}

  \begin{itemize}
  \item Prove theorems by writing code: Coq, Agda, Idris, ...
  \item Prove theorems \textbf{about} programming: \\
    can apply a hundred years of work in formal logic!
  \item Serendipity: Take a thing in \{logic,PL\}, \\ask ``what does this mean in \{PL,logic\}?''
  \item Design of programming language features
  \end{itemize}
\end{frame}

\begin{frame}
  \frametitle{Serendipity}

  What logical concepts have meaning in programming-land?
  \begin{itemize}
  \item Connectives: $\vee$, $\neg$, $\forall$, $\exists$, ...
  \item Properties: Constructivity, consistency, ...
  \item Systems: Modal logic, linear logic, ...
  \end{itemize}
  \vspace{1em}

  What PL concepts have meaning in logic-land?
  \begin{itemize}
  \item \textbf{Evaluation}, Turing-completeness, laziness, mutation, subtyping,
    exceptions, monads, ...
  \end{itemize}

  %% \begin{itemize}
  %% \item Connectives: $\vee$, $\neg$, $\forall$, $\exists$, ...
  %% \item Properties: constructivity, consistency, \\satisfiability, local consistency \& completeness, ...
  %% \item Systems: Modal logic, linear logic, ...
  %% \end{itemize}
  %% What do these mean in programming-land?
  %% \vspace{.2em}

  %% \begin{itemize}
  %% \item Evaluation, turing-completeness, laziness, mutability, subtyping, exceptions, monads.\\
  %% \end{itemize}
  %% What do these mean in logic-land?

\end{frame}

\begin{frame}
  \frametitle{BONUS ROUND: Evaluating our $\lambda$-calculus}

  Read ``$a \steps b$'' as ``$a$ steps to $b$'' or ``whenever you see $a$, you
  can replace it with $b$''.

  \begin{center}
    \begin{tabular}{cc}
      Rule & in Python\\
      \hline
      $\projl \pair{a}{b} \steps a$ & \texttt{(a,b)[0]} evaluates to \texttt{a}\\
      $\projr \pair{a}{b} \steps b$ & \texttt{(a,b)[1]} evaluates to \texttt{b}\\
      \onslide<2->{$(\lambda x. a)\, b \steps \subst{b/x}a$}
      & \onslide<3->{\texttt{(lambda\ x:\ a)(b)} evaluates to}\\
      & \onslide<4>{... well, it's complicated}
    \end{tabular}
  \end{center}

  \onslide<2-> $\subst{b/x} a$ means ``substitute $b$ for $x$ in $a$''.
\end{frame}

\end{document}
