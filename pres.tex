\documentclass{beamer}

\usepackage{proof}

\title{Curry-Howard From The Ground Up}
\author{Michael Arntzenius}
%% TODO: \date

%\usetheme{Malmoe}
%\usecolortheme{seahorse}

\definecolor{ignore}{rgb}{0.8,0.8,0.8}

\newcommand{\isassumed}{\textsf{\tiny{\emph{(assumed)}}}}
\newcommand{\assumed}[1]{\deduce{#1}{\isassumed}}

\begin{document}

\begin{frame}
  \titlepage
\end{frame}

%% \begin{frame}
%%   \frametitle{Table of Contents}
%%   \tableofcontents
%% \end{frame}

\begin{frame}
  \frametitle{What's this talk about?}
  \begin{enumerate}
  \item A simple formal logic (natural deduction)
  \item A simple programming language ($\lambda$-calculus)
  \item How they're \emph{secretly the same thing!!}
    %% TODO: more convincing/accurate
  \item The bigger picture % (``why should I care?'')
  \end{enumerate}
\end{frame}


\section{1. Formal Logic}
%\section{A simple formal logic (natural deduction)}

%\begin{frame} \tableofcontents[currentsection] \end{frame}

\begin{frame}
  %\frametitle{What's a formal logic?}

  \begin{center}
    {\Large\bf What's a formal logic?}
  \end{center}
  %Something that makes it \textbf{perfectly clear} when a proof is valid.

\end{frame}

\newcommand{\rn}[1]{\ensuremath{\mathtt{#1}}}

\begin{frame}
  \frametitle{Inference rules}

  % Inference rules are one way of structuring formal proofs.

  \[
  \infer[\rn{RuleName}]{conclusion}{premise_1 & premise_2 & \cdots & premise_n}
  \]

  \textbf{If} $premise_1$, $premise_2$, ..., and $premise_n$,\\
  \textbf{then} $conclusion$,\\
  \textbf{by} rule $\rn{RuleName}$.
\end{frame}

\begin{frame}
  \frametitle{Defining inference rules}

  % Inference rules can be \textbf{defined}:
  \[
  \infer[\rn{transitivity}]{a < c}{a < b & b < c}
  \]
  This \emph{defines} a rule of inference called \rn{transitivity}.

\end{frame}

\begin{frame}
  \frametitle{Using inference rules in proofs}

  % Once defined, we can use inference rules in \textbf{proofs}:
  % Inference rules can be PLUGGED INTO one another.

  \[
  \infer[\rn{transitivity}]{a < d}{
    \assumed{a < b} &
    \infer[\rn{transitivity}]{b < d}{
      \assumed{b < c} &
      \assumed{c < d}}}
  \]

  This \textbf{proves} that $a < d$ \textbf{from} the (unproven) assumptions $a
  < b$, $b < c$, and $c < d$, \textbf{using} the \rn{transitivity} rule.

\end{frame}

\begin{frame}
  \frametitle{A simple logic}
  \begin{center}
    \begin{tabular}{cc}
      Connective & Meaning\\
      \hline
      $A \wedge B$ & ``$A$ and $B$''\\
      $A \to B$ & ``$A$ implies $B$'', or ``given $A$ then $B$''\\
      \color{ignore} $A \vee B$
      & \color{ignore} ``$A$ or $B$ (\emph{and I know which one})''\\
      \color{ignore} $\neg A$
      & \color{ignore} ``not $A$'' or ``refutation of $A$''
    \end{tabular}
  \end{center}
\end{frame}


\subsection{The $\wedge$ connective}
\begin{frame}
  \frametitle{A simple logic: $\wedge$}

  %This rule lets you prove $A \wedge B$:
  \[
  \infer[\!\wedge\rn{I}]{A \wedge B}{A & B}
  \qquad
  %Given $A \wedge B$, these rules let you prove other things:
  \infer[\!\wedge\rn{E}_1]{A}{A \wedge B} \qquad
  \infer[\!\wedge\rn{E}_2]{B}{A \wedge B}
  \]
\end{frame}

\begin{frame}
  \frametitle{A simple logic: $\wedge$: example proofs}
  \[
  \onslide<2-4>{
    \infer[\!\wedge\rn{E}_1]{B}{
      \infer[\!\wedge\rn{E}_2]{B \wedge C}{
        \assumed{A \wedge (B \wedge C)}}
    }
  }
  \qquad
  \onslide<3-4>{
    \infer[\!\wedge\rn{I}]{A \wedge A}{\assumed{A} & \assumed{A}}
  }
  \qquad
  \onslide<4>{
    \infer[\!\wedge\rn{I}]{B \wedge C}{
      \infer[\!\wedge\rn{E}_2]{B}{\assumed{A \wedge B}}
      & \assumed{C}}
  }
  \]

  {\small For reference:
  \[
  \infer[\!\wedge\rn{I}]{A \wedge B}{A & B} \qquad
  \infer[\!\wedge\rn{E}_1]{A}{A \wedge B} \qquad
  \infer[\!\wedge\rn{E}_2]{B}{A \wedge B}
  \]
  }
\end{frame}


\section{2. Programming Language}
%\section{A simple programming language ($\lambda$-calculus)}

\begin{frame}
  %\frametitle{A $\lambda$-calculus primer}
  TODO: $\lambda$-calculus stuff
\end{frame}


\section{3. Curry-Howard}
%\section{How they're {\it secretly the same thing!!}}

\begin{frame}
  \frametitle{Wait, what?}
  TODO: how are they the same thing?
\end{frame}


\section{4. The Big Picture}
%\section{The bigger picture (``why should I care'')}

\begin{frame}
  \frametitle{In context}
  TODO: what's the bigger picture anyway?
\end{frame}

\end{document}
